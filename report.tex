\documentclass[12pt]{report}

\usepackage{graphics}
\usepackage{fullpage,epsf,graphicx, amstext,url} 

%enable hyperlinks 
\usepackage{hyperref}
\hypersetup{
    colorlinks=true,
    linkcolor=violet,
    filecolor=green,      
    urlcolor=blue,
}

% To get references:
\usepackage[nottoc,numbib]{tocbibind} % To get bibliography into table of contents
\usepackage[backend=biber, style=ieee]{biblatex}
\addbibresource{references.bib}

\begin{document}

\thispagestyle{empty}

%
%	This is a basic LaTeX Template for the TP/MP MSc Dissertation report

\parindent=0pt          %  Switch off indent of paragraphs 
\parskip=5pt            %  Put 5pt between each paragraph  

%	This section generates a title page
%       Edit only the sections indicated to put in the project title, and submission date

\vspace*{0.1\textheight}

\begin{center}
        \huge{\bfseries The Cosmic Dark Matter Density Field}\\
\end{center}

\bigskip

\begin{center}
        \large{Tiernan Stapleton}\\
        \small{\textbf{Supervisor:} Prof. Sadegh Khochfar}\\
        \bigskip
        \large{August 18, 2022}  % Submission Date
\end{center}

\vspace*{0.3\textheight}

\begin{center}
        \includegraphics[width=35mm]{crest.pdf}
\end{center}

\medskip

\begin{center}

\large{
  MSc in Theoretical Physics\\[0.8ex]
  The University of Edinburgh\\[0.8ex]
  2022
}

\end{center}

\newpage


\pagenumbering{roman}

\begin{abstract}

\end{abstract}

\pagenumbering{roman}

\begin{center}
\textbf{Declaration}
\end{center}

\newpage

\begin{center}
\textbf{Personal Statement}
\end{center}

* Lit review
* Work on total mass function
* Work on mass function
* Work on overdensity
* Press Schechter overdensity
* Sigma as func R

\subsubsection*{Example~1: an analytical project}

The project began with an introduction to the spinor-helicity
formalism in four dimensions, with my main source material being
H. Elvang's “Scattering Amplitudes in Gauge Theory and Gravity” [1]. I
read the first chapter, and acquainted myself with the formalism,
and how it worked in a practical sense.

Once I felt more comfortable with it, we moved onto the
six-dimensional spinor-helicity formalism paper, where I spent some
time gaining as strong an understanding of how the formalism worked,
and proving identities.

The next stage was to learn about the generalised unitarity procedure,
with the end goal being to use it to calculate coefficients for some
one loop integral, likely involving massive particles. Learning how
this worked took some time, and proved to be some of the most
difficult material for me to understand.

It wasn't until later that we began to consider applying what I had
learned to a Kaluza-Klein reduction, which ended up being the main
focus of the project. It mixed well with the general theme of
“extra-dimensional theory” the project began with, and allowed me to
apply all that I'd learned and prepared for so far.  The vast majority
of my remaining time was spent calculating coefficients for the scalar
box contribution to the gluon-gluon to two-Kaluza-Klein-particle
amplitude, overcoming a number of problems and errors, to finally have
human-readable, and presentable results.

During the course of the project, I met with my supervisor every week,
in order to discuss my progress and the direction I would head
next. Toward the end, the frequency of our meetings increased
somewhat, as I began to finish my calculations.

I started writing this dissertation in mid-July, and I spent the first
three weeks of August working on it full-time.

Overall, I feel that the project was a success, and I found it to be
extremely enjoyable throughout.


\subsubsection*{Example~2: a computational project}

I spent the first 2 weeks of the project reading the material
surrounding my project - mainly [1] and [2]. I also began to plan out
how I would implement the algorithms in C++, in doing this I gained an
understanding of what the main goals of the first half of my project
would be and how they could be achieved. I identified which Monte
Carlo observables would be useful to measure in these simulations.

For the next 3 weeks I implemented the standard Atlantic City
algorithm and debugged my code whilst developing analysis tools in
python. I compared the results from my simulations to the results from
[3] (for the Random Osculator) and [4] for the EvenMoreRandom
Osculator. Having obtained positive results for the Random Osculator I
started reading up on Heaviside Articulation. I examined how to
integrate a Heaviside Articulator into the simulation in order to
produce the most efficient simulation - the solution I decided on was
to use a package called HeaviArt[5].

Following this I began to integrate the Heaviside Articulator into my
code and test it against the regular algorithm. In addition to this I
ran longer simulations to verify my findings without Articulation.

In mid July I finished implementing Heaviside Articulation into my
code and began looking into how to quantify any improvement in speed
gained by this algorithm. As July progressed I started looking into
how to integrate the EvenMoreRandom Osculator into my code - this was
the most complicated part of the project, as discussed in the body of
this report. Despite much effort on my part, I couldn't get the
results produced by the new algorithm to agree with the old
ones. Following further study of the literature, and long discussions
with Jack O'Bean, it turned out that the original form of Heaviside
Articulation didn't applied to the EvenMoreRandom Osculator. With the
help of Jack and my supervisor, I then developed the new version
described in this report. I also did analytical calculations of the
Four-Point Green-and-White- Function to two orders higher than had
been published previously in the literature.

For the final parts of the summer I worked mainly on perfecting the
algorithm for the Random Osculator and implementing the EvenMoreRandom
Osculators algorithm with the improved Heaviside Articulation. The
final results were encouraging, but more work is clearly needed. To
this end, I have been awarded a studentship by the British University
of Lifelong Learning to extend this work during my PhD Studies at
the non-existent Scottish Highlands Institute of Technology in
Inveroxter.

I started writing this dissertation in mid-July, and I spent the first
three weeks of August working on it full-time.

\newpage

\begin{center}
%\vspace*{2in}
% an acknowledgements section is completely optional but if you decide
% not to include it you should still include an empty {titlepage}
% environment as this initialises things like section and page numbering.
\textbf{Acknowledgements}
\end{center}

\emph{Put your acknowledgements here. Thanking your supervisor for his/her help is standard practice, but it's not compulsory\ldots}

\tableofcontents
\listoftables
\listoffigures

\pagenumbering{arabic}

\chapter{Introduction}

Test \cite{croton_damn_2013}

\chapter{Background}

\chapter{Results and Analysis}

\section{Some results}

\subsection{More results}

\section{Discussion}

\chapter{Conclusions}

\chapter{References and Bibliography}
\printbibliography[heading=none]

\appendix
% the appendix command just changes heading styles for appendices.
\chapter{Appendix}

Some people include in their thesis a lot of detail, particularly lots
of tables containing raw results, figures of intermediate results, or
computer code which no-one will ever read. You should be careful that
anything like this you include should contain some element of
uniqueness which justifies its inclusion.


\end{document}
